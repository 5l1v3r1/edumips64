The GUI of \EM{} is inspired to WinMIPS64 user interface. In fact, the main window
is identical, except for some menus. Please refer to
% chapter~\ref{mips-simulators} for an overview of some MIPS and DLX simulators
% (including WinMIPS64), and to
%\cite{winmips-web} for more information about WinMIPS64.

%In figure~\ref{fig:edumips-main} you can see the main window of \EM{}, composed by
The \EM{} main window is composed by
a menu bar and six frames, showing different aspects of the simulation. There's
also a status bar, that has the double purpose to show the content of memory
cells and registers when you click them and to notify the user that the
simulator is running when the simulation has been started but verbose mode is
not selected. There are more details in the following section.

\index{\EM{}!menu}
\section{The menu bar}
The menu bar contains six menus:

\index{\EM{}!menu!File}
\subsection{File}
The File menu contains menu items about opening files, resetting or shutting
down the simulator, writing trace files.
\begin{itemize}
	\item \textbf{Open...} Opens a dialog that allows the user to choose
	a source file to open.
	\item \textbf{Open recent} Shows the list of the recent files opened by the
	simulator, from which the user can choose the file to open
	\item \textbf{Reset} Resets the simulator, keeping open the file that was
	loaded but resetting the execution.
	\item \textbf{Write Dinero Tracefile...} Writes the memory access data to a
	file, in xdin format.
	\item \textbf{Exit} Closes the simulator.
\end{itemize}
The \textbf{Write Dinero Tracefile...} menu item is only available when a whole
source file has been executed and the end has been already reached.

\index{\EM{}!menu!Execute}
\subsection{Execute}
The Execute menu contains menu items regarding the execution flow of the
simulation.
\begin{itemize}
	\item \textbf{Single Cycle} Executes a single simulation step
	\item \textbf{Run} Starts the execution, stopping when the simulator reaches
	a \texttt{SYSCALL 0} (or equivalent) or a \texttt{BREAK} instruction, or
	when the user clicks the Stop menu item (or presses F9).
	\item \textbf{Multi Cycle} Executes some simulation steps. The number of
	steps executed can be configured through the Setting dialog.
	See~\ref{dialog-settings} for more details.
	\item \textbf{Stop} Stops the execution when the simulator is in ``Run''
	or ``Multi cycle'' mode, as described previously.
\end{itemize}
This menu is only available when a source file is loaded and the end of the
simulation is not reached. The \textbf{Stop} menu item is available only in
``Run'' or ``Multi Cycle'' mode.

\index{\EM{}!menu!Configure}
\subsection{Configure}
The Configure menu provides facilities for customizing \EM{} appearance and
behavior.
\begin{itemize}
	\item \textbf{Settings...} Opens the Settings dialog, described
	in~\ref{dialog-settings}
	\item \textbf{Change Language} Allows the user to change the language used
	by the user interface. Currently only English and Italian are supported.
	This change affects every aspect of the GUI, from the title of the frames to
	the online manual and warning/error messages.
\end{itemize}
The \texttt{Settings...} menu item is not available when the simulator is in
``Run'' or ``Multi Cycle'' mode, because of potential race conditions.

\index{\EM{}!menu!Tools}
\subsection{Tools}
This menu contains only an item, used to invoke the Dinero Frontend dialog.
\begin{itemize}
	\item \textbf{Dinero Frontend...} Opens the Dinero Frontend dialog.
	See~\ref{dialog-dinero}.
\end{itemize}
This menu is not available until you have not executed a program and the
execution has reached its end.

\index{\EM{}!menu!Window}
\subsection{Window}
This menu contains items related to operations with frames.
\begin{itemize}
	\item \textbf{Tile} Sorts the visible windows so that no more that three
	frames are put in a row. It tries to maximize the space occupied by every
	frame.
\end{itemize}
The other menu items simply toggle the status of each frame, making them visible
or minimizing them.

\index{\EM{}!menu!Help}
\subsection{Help}
This menu contains help-related menu items.
\begin{itemize}
	\item \textbf{Manual...} Shows the Help dialog. See~\ref{dialog-help}
	\item \textbf{About us...} Shows a cute dialog that contains the names of
	the project contributors, along with their roles.
\end{itemize}

\index{\EM{}!frames}
\section{Frames}
The GUI is composed by seven frames, six of which are visible by default, and
one (the I/O frame) is hidden.
\index{\EM{}!frames!Cycles}
\subsection{Cycles}
The Cycles frame shows the evolution of the execution flow during time, showing
for each time slot which instructions are in the pipeline, and in which stage of
the pipeline they're located.

\index{\EM{}!frames!Registers}
\subsection{Registers}
The Registers frame shows the content of each register. By left-clicking on them
you can see in the status bar their decimal (signed) value, while
double-clicking on them will pop up a dialog that allows the user to change the
value of the register.

\index{\EM{}!frames!Statistics}
\subsection{Statistics}
The Statistics frame shows some statistics about the program execution.

\index{\EM{}!frames!Pipeline}
\subsection{Pipeline}
The Pipeline frame shows the actual status of the pipeline, showing which
instruction is in which pipeline stage. Different colors highlight different
pipeline stages.

\index{\EM{}!frames!Memory}
\subsection{Memory}
The Memory frame shows memory cells content, along with labels and comments
taken from the source code. Memory cells content, like registers, can be modified
double-clicking on them, and clicking on them will show their decimal value in
the status bar.
The first column shows the hexadecimal address of the memory cell, and the
second column shows the value of the cell. Other columns show additional info
from the source code.

\index{\EM{}!frames!Code}
\subsection{Code}
The Code window shows the instructions loaded in memory. The first column shows
the address of the instruction, while the second column shows the hexadecimal
representation of the instructions. Other columns show additional info taken
from the source code.

\index{\EM{}!frames!Input/Output}
\subsection{Input/Output}
The Input/Output window provides an interface for the user to see the output
that the program creates through the SYSCALLs 4 and 5. Actually it is not 
used for input, as there's a dialog that pops up when a SYSCALL 3 tries to read
from standard input, but future versions will include an input text box.

\index{\EM{}!dialogs}
\section{Dialogs}
Dialogs are used by EduMIPS64 to interact with the user in many ways. Here's a
summary of the most important dialogs:

\index{\EM{}!dialogs!Settings}
\subsection{Settings}
\label{dialog-settings}
In the Settings dialog various aspects of the simulator can be configured.

The Main Settings tab allow to configure forwarding and the number of steps in the
Multi Cycle mode.

The Behavior tab allow to enable or disable warnings during the parsing phase,
the ``Sync graphics with CPU in multi-step execution'' option, when checked,
will synchronize the frames' graphical status with the internal status of the
simulator. This means that the simulation will be slower, but you'll have an
explicit graphical feedback of what is happening during the simulation. If this
option is checked, the ``Interval between cycles'' option will influence how
many milliseconds the simulator will wait before starting a new cycle.
Those options are effective only when the simulation is run using the
``Run'' or the ``Multi Cycle'' options from the Execute menu.

The last two options set the behavior of the simulator when a synchronous
exception is raised. If the ``Mask synchronous exceptions'' option is checked,
the simulator will ignore any Division by zero or Integer overflow exception. If
the ``Terminate on synchronous exception'' option is checked, the simulation
will be halted if a synchronous exception is raised. Please note that if
synchronous exceptions are masked, nothing will happen, even if the termination
option is checked. If exceptions are not masked and the termination option is not
checked, a dialog will pop out, but the simulation will go on as soon as the
dialog is closed. If exceptions are not masked and the termination option is
checked, the dialog will pop out, and the simulation will be stopped as soon as
the dialog is closed.

The last tab allows to change the colors that are associated to the different
pipeline stages through the frames. It's pretty useless, but it's cute.

\index{\EM{}!dialogs!Dinero Frontend}
\subsection{Dinero Frontend}
\label{dialog-dinero}
The Dinero Frontend dialog allows to feed a DineroIV process with the trace file
internally generated by the execution of the program. In the first text box
there is the path of the DineroIV executable, and in the second one there must
be the parameters of DineroIV. 

% Please see~\cite{dinero-web} for further informations about the DineroIV cache simulator.

The lower section contains the output of the DineroIV process, from which you
can take the data that you need.

\index{\EM{}!dialogs!Help}
\subsection{Help}
\label{dialog-help}
The Help dialog contains three tabs with some indications on how to use the
simulator. The first one is a brief introduction to \EM{}, the second one contains
informations about the GUI and the third contains a summary of the supported
instructions.

\index{\EM{}!command line options}
\section{Command line options}
Three command line options are available. They are described in the following
list, with the long name enclosed in round brackets. Long and short names can be
used in the same way.
\begin{itemize}
	\item \texttt{-h (--help)} shows a help message containing the
	simulator version and a brief summary of command line options
	\item \texttt{-f (--file) filename} opens \texttt{filename} in the simulator
	\item \texttt{-d (--debug)} enters Debug mode
\end{itemize}

The \texttt{--debug} flag has the effect to activate Debug mode. In this mode, a
new frame is available, the Debug frame, and it shows the log of internal
activities of \EM{}. It is not useful for the end user, it is meant to be used by
\EM{} developers.

\index{\EM{}!running}
\section{Running \EM{}}
The \EM{} \texttt{.jar} file can act both as a stand-alone executable
\texttt{.jar} file and as an applet, so it can be executed in both ways. Both
methods need the Java Runtime Environment, version 5 or later.

To run it as a stand-alone application, the \texttt{java} executable must be
issued in this way: \texttt{java -jar edumips64-version.jar}, where the
\texttt{version} string must be replaced with the actual version of the
simulator. On some systems, you may be able to execute it by just clicking on
the \texttt{.jar} file.

To embed it in an HTML, the \texttt{<applet>} tag must be used. The \EM{} web
site contains a page that already contains the applet, so
that everyone can execute it without the hassle of using the command line.
